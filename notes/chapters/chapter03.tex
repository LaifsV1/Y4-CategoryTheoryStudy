\documentclass[../main.tex]{subfiles}
\begin{document}

\section{Morphism with no inverse in the category of posets and monotone functions}

Unlike \textbf{Set}, where bijections are isomorphisms, in the category of \textbf{posets and monotone functions}, not every bijection is an isomorphism.

For instance, given the following sets:\par
$A = \{1,2\}$ where $A$ is discrete \par
$B = \{1,2\}$ where $1 \leq 2$ \par
$A \rightarrow B$ could map $2_A \mapsto 1_B$ and $1_A \mapsto 2_B$, which is valid because elements in $A$ are disconnected. \par
$B \rightarrow A$ would not be able to map $1_B \mapsto 2_A$ and $2_B \mapsto 1_A$ (in reverse) and keep order because $2_A \nleq 1_A$

\section{Initial and terminal objects for the category of posets and monotone functions}

\textbf{(Initial object)} \par

When we regard a \textbf{poset} as a category, then the initial object is the \textbf{smallest element}, if it exists. This is because there is a single morphism between the smallest object and any other object; the smallest object is always smaller than any other object.

In the \textbf{category of posets}, however, we have to consider the poset for which there is a unique monotone function between it and every other object.\par

Analogous to \textbf{Set}, this would be the \textbf{empty poset} $(\varnothing , \varnothing)$, which is discrete, indiscrete, and has no relations (is a flat poset). \par

\textbf{(Terminal object)} \par

When we regard a \textbf{poset} as a category, then the terminal object is the \textbf{largest element}, if it exists. This is because there is a single morphism from any object to the largest object; the largest object is always larger than any other object.

In the \textbf{category of posets}, analogous to \textbf{Set}, this would be the \textbf{singleton equality poset} $(1 , =)$. \par

\section{Functors}

A functor $F : \mathcal{C} \rightarrow \mathcal{D}$ is a function that maps between categories $\mathcal{C}$ and $\mathcal{D}$ by mapping all definitions from $\mathcal{C}$ to $\mathcal{D}$ such that:
\begin{enumerate}
  \item all objects in $\mathcal{C}$ are mapped to all objects in $\mathcal{D}$ by $F$ \par
    $\textit{ob}\mathcal{C} \xrightarrow{\quad F\quad } \textit{ob}\mathcal{D}$ \qquad
    i.e. $X \in \mathcal{C}$ then $F(X) \in \mathcal{D}$
  \item all morphisms in $\mathcal{C}$ are mapped to a morphism in $\mathcal{D}$\par
$\forall A,B \in \mathcal{C} \wedge f : A \rightarrow B ~.~ FA \xrightarrow{\quad F_{A,B} f \quad} FB $
  \item the identity morphisms are also mapped and maintained \par
    $\forall A \in \mathcal{C} ~.~ Fid_A = id_{FA}$\par
  \item composition is also mapped and maintained \par
    $\forall A,B,C \in \mathcal{C} \wedge A \xrightarrow{\quad f \quad} B \xrightarrow{\quad g \quad} C ~.~ F(f\underset{A,B}{;}g) = Ff\underset{FA,FB}{;}Fg$ \par
\begin{flalign*}
\begin{diagram}[labelstyle=\scriptscriptstyle]
FA & \rTo{F(f;g)}{}     & FC       \\
  & \rdTo{}{Ff} & \uTo{}{Fg} \\
  &               & FB
\end{diagram} &&
\end{flalign*}
\end{enumerate}
Functors can also be thought of as homomorphisms between categories.

\section{More categories}

\subsection{Monoids}
Any monoid $(M,e,\bullet)$ gives rise to a category $\tilde{M}$ if:
\begin{itemize}
  \item $\textit{ob}\tilde{M} ~\stackrel{\text{def}}{=}~ 1$ \qquad where $1$ is any singleton set. e.g. the unit set $\{()\}$
  \item $\tilde{M}( (), () ) ~\stackrel{\text{def}}{=}~ M$ \qquad morphisms in $() \rightarrow () \in \tilde{M}$ are the elements of $M$
  \item $id_{()} ~\stackrel{\text{def}}{=}~ e$ \qquad the identity morphism is the identity element $e \in M$
  \item $() \xrightarrow{\quad m \quad} () \xrightarrow{\quad m' \quad} () ~\stackrel{\text{def}}{=}~ m \bullet m'$ \qquad composition is the monoid binary relation $(\bullet)$, e.g. concatenation
\end{itemize}

\subsection{Preord and preordered classes}
\textbf{Preord} is the category of preordered sets and monotone maps. \par

Any preordered class (a class can be as big as the universe of all sets) $(A,\leq)$ gives rise to a category $\hat{A}$ if:
\begin{itemize}
  \item objects in $\hat{A}$ are elements in $A$ \par
    $\textit{ob}\hat{A} ~\stackrel{\text{def}}{=}~ A$
  \item the set of morphisms are either the singleton set or the empty set \par
    $\forall x,y \in A ~.~ \hat{A}(x,y) ~\stackrel{\text{def}}{=}~
    \begin{cases}
      1 & x \leq y \\
      \varnothing & x \nleq y
   \end{cases}$
  \item the identity morphism is the element of the singleton set \par $id_{x} ~\stackrel{\text{def}}{=}~ ()$
  \item composition is the element of the singleton set, so all objects are related by this element.\par
    $x \xrightarrow{\quad () \quad} y \xrightarrow{\quad () \quad} z ~\stackrel{\text{def}}{=}~ ()$
\end{itemize}

\subsection{Groupoid}
A \textbf{groupoid} is a category where every morphism is an isomorphism.

The category of sets and bijective functions, for instance, is a \textbf{groupoid}.

\section{Functor examples}

\subsection{Revisiting the inverse of town routes}

To solve the problem for whether any non-identity morphism has an inverse in the category of towns and routes, we can define the following functor:\par

$F : \textbf{Town} \rightarrow \textbf{Nat} = \widetilde{(\mathbb{N},0,+)}$ \qquad i.e. from \textbf{Town} to the monoid of natural numbers
\begin{itemize}
  \item $\forall ~town \in \textbf{Town} ~.~ town \mapsto () \in \textbf{Nat}$
  \item $(f : x_0 \rightarrow x_n = x_0 \xrightarrow{\quad f = [x_0,~\dots~,x_n] \quad} x_n) \mapsto n \in \textbf{Nat}$
  \item $(id_{x_0} = x_0 \xrightarrow{\quad [x_0] \quad} x_0) \mapsto 0 \in \textbf{Nat}$
  \item $(f ; g = x_0 \xrightarrow{\quad f = [x_0,~\dots~,x_n] \quad} x_n  \xrightarrow{\quad g = [x_n,~\dots~,x_m] \quad} x_m) \mapsto n + m \in \textbf{Nat}$
\end{itemize}
Isomorphisms are preserved by functors.\par
Morphisms are equal to the route lengths in \textbf{Nat}. \par
Composition is concatenation of routes in \textbf{Town}, which translates to addition in \textbf{Nat}.\par
An identity morphism is the singleton route in \textbf{Town} and zero $(0)$ in \textbf{Nat}.\par
$\therefore$~ Since adding (composing) any number with a non-zero natural number (morphism) in \textbf{Nat} will never produce the identity $(0)$, and \textbf{Town} is mapped to \textbf{Nat} by $F$ (preserving isomorphisms), concatenating two routes that are not the identity will never produce the identity route either. Thus, no non-identity route has an inverse.

\subsection{Define a functor $C^{op}  \times C \rightarrow $Set---the hom-functor}
Given category $C$, define functor $F : C^{op} \times C \rightarrow \textbf{Set}$ \par

First, we know category $C^{op} \times C$ consists of:
\begin{itemize}
  \item \textbf{(objects)} $(X,Y) \in C^{op} \times C$ where $X \in C^{op}$ and $Y \in C$. \par
    i.e.  $X,Y \in C$ since objects in $C$ are in $C^{op}$
  \item \textbf{(morphisms)} $(X,Y) \mapsto (X',Y') \in C^{op} \times C$ is $(f,g)$ where \par
    $f : X \rightarrow X' \in C^{op}$ \par
    $g : Y \rightarrow Y' \in C$
  \item \textbf{(identities)} $id_{(X,Y)} = (id_X , id_Y)$
  \item \textbf{(composition)} $(f,g) ; (f',g') = (f;f',g;g')$ where\par
    $f;f' \in C^{op}$ \quad or \quad $f';f \in C$ \par
    $g;g' \in C$
\end{itemize}

Now we define what we map to in \textbf{Set}:
\begin{itemize}
  \item \textbf{(objects)} $(X,Y) \in (C \times C^{op}) \mapsto hom_C(X,Y) \in \textbf{Set}$
  \item \textbf{(morphisms)} for \textbf{Set}, a morphism is all functions $hom_C(X,Y) \rightarrow hom_C(X',Y')$ \par
    Given $(f,g)$ where \par
    $f : X \rightarrow X' \in C^{op}$ \qquad i.e. $f : X' \rightarrow X \in C$ \par
    $g : Y \rightarrow Y' \in C$ \par
    We can define a morphism (functions) $hom_C(X,Y) \rightarrow hom_C(X',Y') \in \textbf{Set}$ as follows: \par
    if $h \in hom_C(X,Y)$ \par
    then $f;h;g \in hom_C(X',Y')$ \par
    so $h \mapsto f;h;g$
\end{itemize}
In short:
\begin{tabular}{cccl}
\begin{diagram}[labelstyle=\scriptscriptstyle]
(X,Y)    &\quad hom_C(X,Y)    & \quad h \in hom_C(X,Y)  \\
\uTo{f}\dTo{g}& \dTo        & \dMapsto       \\
(X',Y')  &\quad hom_C(X',Y')  & \quad f;h;g \in hom_C(X',Y')
\end{diagram}
& &
\end{tabular}

Now we check it is a valid functor:
\begin{itemize}
  \item \textbf{(identities)} $(id_X,id_Y) \mapsto h \mapsto id_X ; h ; id_Y = id_{hom_C(X,Y)}$
  \item \textbf{(composition)} Given $h \in hom_C(X,Y)$ \par
We want to show that the following composition is valid: \par
$hom_C(X,Y) \xrightarrow{A} hom_C(X',Y') \xrightarrow{B} hom_C(X'',Y'')$
\par
    $h' = f;h;g$ and $f';h';g' = f';f;h;g;g'$\par
    we know that composition $A ; B : hom_C(X,Y) \rightarrow hom_C(X'',Y'') = f';f;h;g;g'$\par
We now prove composition is maintained by the functor:\par
    Given $(f,g) = (X' \rightarrow X'',Y \rightarrow Y')$ and $(f',g') = (X \rightarrow X',Y' \rightarrow Y'')$\par
$\newline$
$F((f;g)~;~(f';g'))$\par $= F(f';f~,~g;g')$ \par
$=(f';f;h;g;g')$ where $h \in hom_C(X,Y)$ \par
$= f';(f;h;g);g'$\par
$= f'; F(f,g) ; g'$\par
$= F(f,g) ; F(f';g)$\par

$\newline$
i.e.

\begin{tabular}{cccl}
\begin{diagram}[labelstyle=\scriptscriptstyle]
hom_C(X,Y)          &                & \\
                    & \rdTo{f;h;g}   & \\
\dTo{(f';f);h;(g;g')}{} &            & hom_C(X',Y') \\
              & \ldTo{f';(f;h;g);g'} & \\
hom_C(X'',Y'')      &                &
\end{diagram}
& &
\end{tabular}

\end{itemize}

\section{Homework exercises}

\subsection{Define a functor from $C$ to a preorder}

Every category can be turned into a preordered class.

Given category $C$, write a preorder $X \leq Y$ when $\exists X \rightarrow Y$

\textbf{(Part 1) Prove $\leq$ is a preorder on $obC$:}

A preorder is a reflexive and transitive binary relation $\leq$.
\begin{itemize}
\item reflexive because for any $X \in C$, $id_X$ is a morphism $X \rightarrow X$, so $X \leq X$ exists and is reflexive.
\item transitive because if $X \leq Y \leq Z$ then we can pick a map $f : X \rightarrow Y$ and a map $g : Y \rightarrow Z$ so $f;g$ is a map $X \rightarrow Z$. This morphism is mapped to $X \leq Y$, which means the composition is transitive.
\end{itemize}

For instance, in \textbf{Set}, $A \leq B$ if $B$ is non-empty or $A$ is empty.

(functions are left-total and single-valued)

\textbf{(Part 2) Define a functor $C \rightarrow \widehat{C} = \widehat{(\textit{ob}C, \leq)}$}

$F : C \rightarrow \widehat{obC,\leq}$

\begin{itemize}
  \item $\textit{ob} \widehat{C} \stackrel{\text{def}}{=}  \textit{ob}C$ \qquad i.e. $\forall X \in C ~.~ X \in C \mapsto X \in \widehat{C}$

  \item Given $X \xrightarrow{f} Y \in C$, we know $\widehat{C}(X,Y) \stackrel{\text{def}}{=} 1$ \qquad
    i.e. $\forall X \leq Y \in C ~.~ f \mapsto ()$

therefore:
\begin{tabular}{l}
\begin{diagram}
A       &          & FA = A  \\
\dTo{F} & \rMapsto & \dTo{}{()}\\
B       &          & FB = B
\end{diagram}
\end{tabular}
\end{itemize}

We know this is valid because:
\begin{itemize}
  \item $id_X \stackrel{\text{def}}{=} ()$ \qquad i.e. $\forall x,id_X \in C ~.~ id_X \mapsto ()$
  \item $X \xrightarrow{\quad () \quad} Y \xrightarrow{\quad () \quad} Z ~\stackrel{\text{def}}{=}~ ()$ \qquad i.e. composition is maintained trivially,
\end{itemize}

Additionally, since this definition describes a category that rises from a preordered class (section 3.4.2), we can argue from the resulting category that $C$ is a preordered class.

\subsection{Full and faithful}

Given $F_{X,Y} : hom_C(X,Y) \rightarrow hom_D(F(X),F(Y))$

\begin{itemize}
\item A \textbf{full} mapping is one which is \textbf{surjective}. (left-total)

So $F_{X,Y}$ is full if it is surjective.

i.e. every morphism in $F~C$ is mapped to by at least one morphism of $C$.

symbolically: $\forall g \in F~C ~.~\exists f \in C ~.~g =F(f)  $

\item A \textbf{faithful} mapping is one which is \textbf{injective}.

*note: injective and single-valued are not the same.

i.e. $F_{X,Y}$ is faithful when for every $X,Y \in C$, $f : X \rightarrow Y$ and $g : X \rightarrow Y$, $F~f = F~g$ implies $f = g$

in short: $\forall f , g \in C~.~F(f) = F(g) \Rightarrow f = g$


\item $F_{X,Y}$ is \textbf{fully faithful} if it is both full and faithful.

\end{itemize}

Note that these definitions only care about morphisms being mapped.

Is the functor defined in 3.6.2 faithful and/or full?

We check:
\begin{itemize}

\item The functor is full because:

$\forall g : F X \rightarrow F Y \in D$

\quad $\exists f : X \rightarrow Y \in C$

\qquad such that $F_{X,Y}(f) = g$

since $g$ impies $X \leq Y$ and $X \leq Y$ implies there is an $f : X \rightarrow Y$, then $f \mapsto g$, so every $f$ is mapped.

\item The functor is not faithful for $C$ if $C(X,Y) = \{f,g,h\}$. This is because all morphisms will be mapped to the same thing, the empty tuple.

So, $f,g,h \mapsto ()$, meaning, $F_{X,Y}$ is not single valued. However, is this the case for every $C$?

To prove this is not the case, we can define a category $D$ where this is an injective mapping. e.g. Given $hom_D(X,Y) = {k}$, $k \mapsto ()$. This is injective.

Therefore, $F$ is \textbf{not faithful in general}.

\end{itemize}

\subsection{Prove category $(C(X,X), id_X, ;_{X,X})$ is a monoid}

Given category $C$ with object $X$, the monoid $M = (C(X,X), id_X, \underset{X,X};)$ is defined as follows:
\begin{itemize}
  \item $id_X$ is the identity element for morphisms
  \item $;_{X,X}$ (composition) is an associative binary operator by definition (it comes from a category, where--by definition--composition is associative)
\end{itemize}

%%%%%%%%%%%%%%%%%%%%%%%%%%%%%%%%%%%%%%%%%%%%%%%%%%%%%%%%%%%%%%%%%%%%%%%%%%%%%%%%%%%%%%%%%%%%%%%%%%%%%%%%%%%%%%%%%%%%%%%%%%%%%%%%%%%%%%%%%%%%%%%%%%%%%%%%%%%%%%%%%%%%%%%%%%%%%%%%%%%%%%%%%
% \subsection{Define a functor $\widehat{(C(X,X), id_X, ;_{X,X})}$ to $C$}                                                                                                              %
%                                                                                                                                                                                       %
% $F : \widehat{(C(X,X), id_X, ;_{X,X})} \rightarrow C$                                                                                                                                 %
%                                                                                                                                                                                       %
% Given $C(X,X) = {()}$                                                                                                                                                                 %
%                                                                                                                                                                                       %
%                                                                                                                                                                                       %
% \begin{itemize}                                                                                                                                                                       %
% \item first we map objects: $F_{objects} : (C(X,X)) \rightarrow obC = () \mapsto X$                                                                                                   %
%                                                                                                                                                                                       %
% Objects in the category that rises from a monoid are the singleton set.                                                                                                               %
%                                                                                                                                                                                       %
% \item then we map morphisms: $F_{morphisms} : C(X,X) \rightarrow C(X,X) = f \mapsto f$                                                                                                %
%                                                                                                                                                                                       %
% Morphisms in the category that rises from a monoid are in the carrier set, i.e. $C(X,X)$. So we have all morphisms available.                                                         %
%                                                                                                                                                                                       %
% \end{itemize}                                                                                                                                                                         %
%                                                                                                                                                                                       %
% This maintains composition and identity by definition of $C$, therefore, it is valid. This is a fully faithful functor since all morphisms are mapped to $C$, and it's single valued. %
%                                                                                                                                                                                       %
% However, we could also define it this way:                                                                                                                                            %
%                                                                                                                                                                                       %
% \begin{itemize}                                                                                                                                                                       %
% \item we map objects: $F_{objects} : (C(X,X)) \rightarrow obC = () \mapsto X$                                                                                                         %
%                                                                                                                                                                                       %
% \item we map morphisms: $F_{morphisms} : C(X,X) \rightarrow C(X,X) = f \mapsto id_x$                                                                                                  %
% \end{itemize}                                                                                                                                                                         %
%                                                                                                                                                                                       %
% This is perfectly valid, however, the result is not faithful in general and not full in general either.                                                                               %
%                                                                                                                                                                                       %
% \subsection{The isomorphisms of a category form a groupoid}                                                                                                                           %
%                                                                                                                                                                                       %
% Given a category $C$, show that its isomorphisms form a groupoid.                                                                                                                     %
%                                                                                                                                                                                       %
% A groupoid $(G,^{-1},*)$ satifies the following conditions:                                                                                                                           %
% \begin{itemize}                                                                                                                                                                       %
%   \item \textbf{all defined relations are associative}: $\forall a*b,b*c \in C ~.~ (a * b) * c = a * (b * c)$                                                                         %
%   \item \textbf{inverses are defined}: $\forall a \in C ~.~ (a^{-1} * a) \wedge (a * a^{-1})$                                                                                         %
%   \item \textbf{identities are defined}: $\forall a * b \in C ~.~ (a * b * b^{-1} = a) \wedge (a^{-1} * a * b = b)$                                                                   %
% \end{itemize}                                                                                                                                                                         %
% It follows that:                                                                                                                                                                      %
% \begin{itemize}                                                                                                                                                                       %
%   \item $(a^{-1})^{-1} = a$ is always defined, every morphism is an isomorphism.                                                                                                      %
%   \item $\forall a*b \in C ~.~ (a * b)^{-1} =  a^{-1} * b^{-1}$                                                                                                                       %
% \end{itemize}                                                                                                                                                                         %
%                                                                                                                                                                                       %
% Proof: \par                                                                                                                                                                           %
% We attempt to define a groupoid $(C(X,Y),^{-1},;_{X,Y})$ for all ismorphisms $X \cong Y \in C$.                                                                                       %
% \begin{itemize}                                                                                                                                                                       %
% \item Morphism composition is associative, meaning, ismorphism composition is also associative.\par                                                                                   %
% All relations in the groupoid of isomorphisms is defined by morphism composition.\par                                                                                                 %
% Therefore, \textbf{all defined relations are associative}.                                                                                                                            %
% \item All isomorphisms in $C$ are morphisms with inverses. \par                                                                                                                       %
% $\forall A \cong B \in C ~.~ f : A \rightarrow B \wedge f^{-1} : B \rightarrow A$ \par                                                                                                %
% Compositions of isomorphisms therefore have inverses.\par                                                                                                                             %
% Therefore, \textbf{inverses are defined} in $(C(X,Y),^{-1},;)$.                                                                                                                       %
%                                                                                                                                                                                       %
% \item All objects in $C$ have a defined identity morphism. \par                                                                                                                       %
% All isomorphisms in $C$ include identity morphisms.\par                                                                                                                               %
% Therefore, \textbf{identities are defined} in $(C(X,Y),^{-1},;)$.                                                                                                                     %
% \end{itemize}                                                                                                                                                                         %
%                                                                                                                                                                                       %
% Given $(C(X,Y),^{-1},;_{X,Y})$ formed by all isomorphisms of $C$ satisfy the definition of a groupoid,  $(C(X,Y),^{-1},;_{X,Y})$ is a groupoid; isomorphisms of $C$ form a groupoid.  %
%%%%%%%%%%%%%%%%%%%%%%%%%%%%%%%%%%%%%%%%%%%%%%%%%%%%%%%%%%%%%%%%%%%%%%%%%%%%%%%%%%%%%%%%%%%%%%%%%%%%%%%%%%%%%%%%%%%%%%%%%%%%%%%%%%%%%%%%%%%%%%%%%%%%%%%%%%%%%%%%%%%%%%%%%%%%%%%%%%%%%%%%%

\end{document}