\documentclass[../main.tex]{subfiles}
\begin{document}
\section{Definition 1--Using natural transformations}

Let $\mathcal{C}$ be a category.

A monad on $\mathcal{C}$ consists of:
\begin{itemize}
\item a functor $T : \mathcal{C} \rightarrow \mathcal{C}$ (endofunctor)
\item a natural transformation $\eta$ called the unit:

\begin{tabular}{cccl}
\begin{diagram}
\mathcal{C}&\rTo{id} &\mathcal{C}\\
           &\dImplies{\eta} &\\
\mathcal{C}&\rTo{T}  &\mathcal{C}
\end{diagram}
\end{tabular}

\item a natural transformation $\mu$ called multiplication:

\begin{tabular}{cccl}
\begin{diagram}
\mathcal{C}&\rTo{T}&\mathcal{C}&\rTo{T}&\mathcal{C}\\
           & &\dImplies{\mu}& &\\
\mathcal{C}& &\rTo{T}&  &\mathcal{C}
\end{diagram}
\end{tabular}

\end{itemize}

satisfying three properties:

\begin{itemize}
\item the right identity law

\begin{tabular}{cccl}
\begin{diagram}
T&\rTo{\eta T} &T^2\\
 &\rdTo{id}    &\dTo{}{\mu}\\
 &  &T
\end{diagram}
\end{tabular}

\item the left identity law

\begin{tabular}{cccl}
\begin{diagram}
T&\rTo{T \eta} &T^2\\
 &\rdTo{id}    &\dTo{}{\mu}\\
 &  &T
\end{diagram}
\end{tabular}

\item the associativity law

\begin{tabular}{cccl}
\begin{diagram}
T^3 & \rTo{\mu T} & T^2 \\
\dTo{T \mu} & & \dTo{}{\mu} \\
T^2& \rTo{}{\mu} & T
\end{diagram}
\end{tabular}
\end{itemize}

It is similar to a monoid, so the name is not a coincidence:
\begin{itemize}
\item set $X$
\item element $e \in X$
\item function $* : X^2 \rightarrow X$
\end{itemize}

Also satisfying three equations:
\begin{itemize}
\item $\forall x \in X . x * e = x$
\item $\forall x \in X . e * x = x$
\item $\forall x,y,z \in X . (x*y)*z = x*(y*z)$
\end{itemize}

\section{Definition 2--As a Kleisli triple}

Let $\mathcal{C}$ be a category.

A Kleisli triple on $\mathcal{C}$ consists of:
\begin{itemize}
\item for each object $A$, we have an object $TA$ and a morphism $\eta_A : A \rightarrow TA$.
\item for objects $A,B$ and morphism $f : A \rightarrow TB$, we have a morphism $f* : TA \rightarrow TB$.
\end{itemize}
such that the following equations are satisfied:
\begin{itemize}
\item left identity law:

\begin{tabular}{cccl}
\begin{diagram}
A &\rTo{f} & TB \\
  &\rdTo(1,2){}{\eta}\ruTo(1,2){}{f*_{A,B}} &  \\
  & TA      &
\end{diagram}
\end{tabular}

analogous to:

$a : A , f : A \rightarrow TB \vdash~ {\tt return ~} a >>= f = f~a$

in Haskell.

\item right identity law: \quad $\eta_B *_{B,B} = id_{TB}$

analogous to:

$p : TB \vdash p >>= \lambda x . {\tt ~return~} x = p : TB$

in Haskell.

\item associativity law:

\begin{tabular}{cccl}
\begin{diagram}
TA &\rTo{f*} & TB \\
  &\rdTo{}{(f;g)*} & \dTo{g*}  \\
  &  & TC
\end{diagram}
\end{tabular}

analogous to:

$p : TA , f : A \rightarrow TB , g : B \rightarrow TC \\
\vdash (p >>= f) >>= g = p >>= (f >>= g)$

in Haskell.

\end{itemize}

Even though the Kleisli triple makes no mention of naturality, it is equivalent to the previous definition given.

\section{Example: Maybe and Exception monad}

$Maybe~X = X + 1 = \{inl x | x \in X\} \cup \{inr ()\}$

$Exc_E~X = X + E$ where E is a set of behaviours.

The $Exc$ (Exception monad) generalises $Id$ (Identity monad) and $Maybe$ monad:
\begin{itemize}
\item $Id$: $E = {} = \varnothing$
\item $Maybe$: $E = 1 = \{()\}$
\end{itemize}

In \textbf{Set}, where morphisms are functions, let's check that $Exc_E$ is indeed a monad.

\begin{enumerate}

\item First we map functions to $Exc_E$:

\begin{tabular}{cccl}
\begin{diagram}
X       &          & X + E \\
\dTo{f} & \rMapsto & \dTo \\
Y       &          & Y + E
\end{diagram}
\end{tabular}
\qquad
\begin{tabular}{cccl}
\begin{diagram}
inl~x   &  & inr~e \\
\dMapsto&  &\dMapsto\\
inl~f x &  & inr~e
\end{diagram}
\end{tabular}

\item Unit at $X$ (the return) is

\begin{tabular}{cccl}
\begin{diagram}
X & \rTo & X + E \\
x & \rMapsto & inl~x
\end{diagram}
\end{tabular}

We can check that this is natural (as in definition 1):

for
\begin{tabular}{cccl}
\begin{diagram}
X       &  & X & \rTo{inl} & X + E \\
\dTo{f} &  & \dTo{f} &  & \dTo{}{Tf}\\
Y       &  & Y & \rTo{inl} & Y + E
\end{diagram}
\end{tabular}

therefore we have a single case

\begin{tabular}{cccl}
\begin{diagram}
x           & \rMapsto{inl} & inl~x \\
\dMapsto{f} &               & \dMapsto{}{Tf}\\
f x         & \rMapsto{inl} & inl~f~x
\end{diagram}
\end{tabular}
which does commute

\qquad so
\begin{tabular}{cccl}
\begin{diagram}
X       & \rTo{\eta_X} & TX \\
\dTo{f} &              & \dTo{}{Tf}\\
Y       & \rTo{\eta_Y} & TY
\end{diagram}
\end{tabular}

\item Multiplication $(T^2 X \xrightarrow{\mu_X} TX)~\mu$ is:

$\mu_X : (X + E) + E \rightarrow X + E$

\begin{tabular}{cccl}
\begin{diagram}
inl~inl~x & \rMapsto &inl~x
\end{diagram}
\\
\begin{diagram}
inl~inr~e & \rMapsto &inr~e
\end{diagram}
\\
\begin{diagram}
inr~e     & \rMapsto &inr~e
\end{diagram}
\end{tabular}

Once again, we can check it is natural:

for
\begin{tabular}{cccl}
\begin{diagram}
X       &        & (X+E)+E & \rTo{ \quad\mu_X \quad} & X + E \\
\dTo{f} & \qquad\qquad & \dTo{(f + E)+E} &  & \dTo{}{f + E}\\
Y       &        & (Y+E)+E & \rTo{ \quad\mu_Y \quad} & Y + E
\end{diagram}
\end{tabular}

we have to check 3 cases:
\begin{enumerate}
\item
\begin{tabular}{cccl}
\begin{diagram}
inl~inl~x   & \rMapsto{\mu_x} & inl~x \\
\dMapsto{f} &                 & \dMapsto{}{f}\\
inl~inl~fx  & \rMapsto{\mu_x} & inl~fx
\end{diagram}
\end{tabular}

\item
\begin{tabular}{cccl}
\begin{diagram}
inl~inr~e   & \rMapsto{     } & inr~e \\
\dMapsto{ } &                 & \dMapsto{}{ }\\
inl~inr~e   & \rMapsto{     } & inr~e
\end{diagram}
\end{tabular}

\item
\begin{tabular}{cccl}
\begin{diagram}
    inr~e   & \rMapsto{     } & inr~e \\
\dMapsto{ } &                 & \dMapsto{}{ }\\
    inr~e   & \rMapsto{     } & inr~e
\end{diagram}
\end{tabular}

\end{enumerate}
which all commute.

\item Finally, we have to check the three properties:

\begin{enumerate}
\item left identity law:

\begin{tabular}{cccl}
\begin{diagram}
T&\rTo{\eta T} &T^2\\
 &\rdTo{id}    &\dTo{}{\mu}\\
 &  &T
\end{diagram}
\end{tabular}

i.e. \quad $\forall X \in C$
\begin{tabular}{cccl}
\begin{diagram}
TX&\rTo{(\eta T)_X} &T^2X\\
 &\rdTo{id_X}    &\dTo{}{\mu}\\
 &  &TX
\end{diagram}
\end{tabular}

This can be defined using left-whiskering:

\begin{tabular}{cccl}
\begin{diagram}
B& \rTo{H}&C&\rTo{F}            &D\\
 &        & &\dImplies{}{\alpha}& \\
B& \rTo{H}&C&\rTo{G}            &D\\
\end{diagram}
\end{tabular}
therefore
\begin{tabular}{cccl}
\begin{diagram}
 &        &FHX\\
X&\rMapsto&\dTo{}{\alpha_{HX}}\\
 &        &GHX\\
\end{diagram}
\end{tabular}

$\therefore$
\begin{tabular}{cccl}
\begin{diagram}
TX&\rTo{\eta_{TX}} &T^2X\\
 &\rdTo{id_X}    &\dTo{}{\mu}\\
 &  &TX
\end{diagram}
\end{tabular}

\item right identity law:

\begin{tabular}{cccl}
\begin{diagram}
TX&\rTo{(T\eta)_X} &T^2X\\
 &\rdTo{id_X}    &\dTo{}{\mu}\\
 &  &TX
\end{diagram}
\end{tabular}

This can be defined using right-whiskering:

\begin{tabular}{cccl}
\begin{diagram}
C&\rTo{F}            &D& \rTo{K}&E \\
 &\dImplies{}{\alpha}& & \\
C&\rTo{G}            &D& \rTo{K}&E \\
\end{diagram}
\end{tabular}
therefore
\begin{tabular}{cccl}
\begin{diagram}
 &        &KFX\\
X&\rMapsto&\dTo{}{K \alpha_X}\\
 &        &KGX\\
\end{diagram}
\end{tabular}

$\therefore$
\begin{tabular}{cccl}
\begin{diagram}
TX&\rTo{T\eta_{X}} &T^2X\\
 &\rdTo{id_X}    &\dTo{}{\mu}\\
 &  &TX
\end{diagram}
\end{tabular}

\item associativity law:

\begin{tabular}{cccl}
\begin{diagram}
T^3X& \rTo{\mu_{TX}} & T^2X \\
\dTo{T \mu_X} & & \dTo{}{\mu_X} \\
T^2X& \rTo{}{\mu_X} & TX
\end{diagram}
\end{tabular}
\end{enumerate}

Now the actual proof that the properties indeed hold:

\begin{enumerate}
\item There are two cases to show the left-identity law; $inl~x$ and $inr~e$:

\begin{tabular}{cccl}
\begin{diagram}
inl~x&\rMapsto{inl} &inl~inl~x\\
 &\rdMapsto{id}    &\dMapsto{}{\mu}\\
 &  &inl~x
\end{diagram}
\end{tabular}
\qquad and \qquad
\begin{tabular}{cccl}
\begin{diagram}
inr~e&\rMapsto{inl} &inl~inr~e\\
 &\rdMapsto{id}    &\dMapsto{}{\mu}\\
 &  &inr~e
\end{diagram}
\end{tabular}

\item There are two cases to show the right-identity law; $inl~x$ and $inr~e$:

\begin{tabular}{cccl}
\begin{diagram}
inl~x&\rMapsto{T~inl} &inl(inr~x)\\
 &\rdMapsto{id}    &\dMapsto{}{\mu}\\
 &  &inl~x
\end{diagram}
\end{tabular}
\qquad and\qquad
\begin{tabular}{cccl}
\begin{diagram}
inr~e&\rMapsto{T~inl} &inl(inr~e)\\
 &\rdMapsto{id}    &\dMapsto{}{\mu}\\
 &  &inr~e
\end{diagram}
\end{tabular}

\item There are four cases to show the associativity law; $inl~inl~inl~x$, $inl~inl~inr~e$, $inl~inr~e$, and $in~e$:

\begin{tabular}{cccl}
\begin{diagram}
inl~inl~inl~x & \rTo{\mu_{X+E}} & inl~inl~x \\
\dTo{\mu_X + E} & & \dTo{}{\mu_X} \\
inl(inl~x)& \rTo{}{\mu_X} & inl~x
\end{diagram}
\end{tabular}
\qquad \qquad
\begin{tabular}{cccl}
\begin{diagram}
inl~inl~inr~e & \rTo{\mu_{X+E}} & inl~inr~e \\
\dTo{\mu_X + E} & & \dTo{}{\mu_X} \\
inl~inl~e& \rTo{}{\mu_X} & inr~e
\end{diagram}
\end{tabular}

\begin{tabular}{cccl}
\begin{diagram}
inl~inl~e & \rTo{\mu_{X+E}} & inr~e \\
\dTo{\mu_X + E} & & \dTo{}{\mu_X} \\
inl~inr~e& \rTo{}{\mu_X} & inr~e
\end{diagram}
\end{tabular}
\qquad \qquad
\begin{tabular}{cccl}
\begin{diagram}
inr~e & \rTo{\mu_{X+E}} & inr~e \\
\dTo{\mu_X + E} & & \dTo{}{\mu_X} \\
inr~e& \rTo{}{\mu_X} & inr~e
\end{diagram}
\end{tabular}
\end{enumerate}

With this, we have shown that the properties hold, and that $Exc_E$ is indeed a monad.

\end{enumerate}

\section{Kleisli extension}

Monads are many times implemented using the Kleisli extension. As with Haskell.

$ (-)^* : hom(X,TY) \rightarrow hom(TX,TY)$

Given a monad $<T,\eta,\mu>$ over category $C$ and a morphism $f : X \rightarrow TY$:

$f^* : TX \rightarrow TY\\
 f^* = \mu_Y \circ Tf$

In other words:

$f : X \rightarrow TY$ is any function that gives us the monad type in Haskell. This also matches $\eta$, which is equivalent to the {\tt return} function.

$f^* : TX \rightarrow TY$ is analogous to the bind function in Haskell. The difference is that arguments are in different order.

For example, if we apply this to the $Exc_E$ monad:

$X \xrightarrow{\quad f\quad } Y + E \\
 X+E \xrightarrow{\quad f^*\quad } Y + E$

so

$inl~x  \mapsto f~x\\
 inr~e \mapsto inr~e$

\section{Comparison with Haskell}

Monads can be defined in the standard mathematical way or through the Kleisli triple. We can either have:
\begin{itemize}
\item $<T,\eta,\mu>$, where $T$ gives us the type of the monad, $\eta$ is the {\tt return} and $\mu$ is the {\tt join}. Note that $T$ is a functor, so in Haskell, this would be a pair $(T,fmap)$, which maps objects and functions.
\item $<T,\eta,(-)^*>$, where $T$ gives us the type, again a pair with {\tt fmap}, $\eta$ is the {\tt return}, and $(-)^*$ is equivalent to {\tt bind} (>>=).
\end{itemize}

This can be seen more clearly with the types:

{\tt join} and $\mu$ (multiplication):

\qquad {\tt join :: T T A -> T A}

\qquad $\mu : T T A \rightarrow T A$

{\tt bind} and $(-)^*$ (Kleisli extension operator):

\qquad {\tt >>= :: T A -> (A -> T B) -> T B}

\qquad $(-)^* : (A \rightarrow T B) \rightarrow (T A \rightarrow T B)$

We can show that a Kleisli triple on $C$ is equivalent to the standard mathematical definition of a monad on $C$ because:
\begin{itemize}
\item $\mu$ and $\eta$ give you {\tt >>=}, i.e. $(-)^*$
\item $(-)^*$ and $\eta$ give you {\tt join}, i.e. $\mu$
\end{itemize}

\section{Exercise: State monad}

Let $S$ be a set (for some state, i.e. the type of the state/store).

We can define a monad on \textbf{Set} with $T~A = S \rightarrow (S \times A)$

In programming, $T~A$ would be a stateful program that returns a value of type $A$.

For instance, if $A = \mathbb{N}$, then $T~\mathbb{N} = S \rightarrow (S \times \mathbb{N})$.

We could write a program that uses this state:

\qquad $\lambda b : {\tt bool} . ({\tt not}~b,b) : T~{\tt bool}$

This program would take the original state, invert it, and then output the original state.

With that context, we can define the monad either the standard way, or as a Kleisli triple. Defining it as a Kleisli triple:

Let $State$ be the monad:
\begin{itemize}
\item $State~X = S \rightarrow (S \times X)$
\item The Unit ($\eta$)

$\eta : X \rightarrow (S \rightarrow (S \times X))$

$\eta = x \mapsto (s \mapsto (s,x))$ where $s : S$ and $x : X$

\item The Kleisli operator (equivalent to bind):

Given $g : X \rightarrow (S \rightarrow (S \times Y))$

$g^* : (S \rightarrow (S \times X)) \rightarrow (S \rightarrow (S \times Y))$

$g* = f \mapsto s \mapsto (g (\pi_{2}~ f~ s))~ (\pi_{1} ~f~ s)$

\end{itemize}

This is analogous to the Haskell state monad. Given $State$ is analogous to our functor $T$:

{\tt newtype State s a = State { runState :: s -> (a, s) }}

{\tt return :: a -> State s a \\
return x = State ( \textbackslash s -> (x, s) )}

{\tt (>>=) :: State s a -> (a -> State s b) -> State s b \\
(State h) >>= f \\
= State ( \textbackslash s -> let (a,new_sate) = h s in f a new_state )
}

Alternatively:

{\tt type State s a = s $\rightarrow$ (a,s)\\
return x = $\lambda$ s $\rightarrow$ (x,s)\\
f >>= g  ~~= $\lambda$ s $\rightarrow$ \textbf{case} f s \textbf{of} (x,s') $\rightarrow$ g x s' }



\end{document}