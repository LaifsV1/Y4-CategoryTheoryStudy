\documentclass[../main.tex]{subfiles}
\begin{document}

\section{Small categories}

To define a small category, we will need the following definition:

\qquad
\begin{tabular}{lll}
  \tabitem Ob  &: Set &Set of all objects \\
  \tabitem Hom &: Ob  $\rightarrow$ Ob  $\rightarrow$ Set &Set of all morphisms\\
  \tabitem id  &:  $\forall$ X : Ob .Hom X X &Identity morphism\\
  \tabitem _;_ &:  $\forall$ X, Y, Z : Ob .\\
               &\quad Hom X Y $\rightarrow$ Hom Y Z $\rightarrow$ Hom X Z  &Composition of morphisms\\
\end{tabular}

and the following laws:

\qquad
\begin{tabular}{lll}
  \tabitem left-identity law\\
  \tabitem right-identity law\\
  \tabitem associativity law\\
\end{tabular}

This can be implemented in Agda:

\begin{tabular}{lll}
\toprule
{\tt
\begin{lstlisting}[mathescape]
record Cat : Set$_1$ where
  field Ob  : Set
        Hom : Ob $\rightarrow$ Ob $\rightarrow$ Set
        id  : {X : Ob} $\rightarrow$ Hom X X
        _$\blacktriangleright$_ : {X Y Z : Ob} $\rightarrow$ Hom X Y $\rightarrow$ Hom Y Z $\rightarrow$ Hom X Z

        left_id  : {X Y : Ob} {f : Hom X Y} $\rightarrow$ id $\blacktriangleright$ f $\equiv$ f
        right_id : {X Y : Ob} {f : Hom X Y} $\rightarrow$ f $\blacktriangleright$ id $\equiv$ f
        assoc    : {W X Y Z : Ob}
                   {f : Hom W X}
                   {g : Hom X Y}
                   {h : Hom Y Z} $\rightarrow$ f $\blacktriangleright$ (g $\blacktriangleright$ h) $\equiv$ (f $\blacktriangleright$ g) $\blacktriangleright$ h
\end{lstlisting}
}
\\
\bottomrule
\end{tabular}

With this, an instance of any small category can be made.

For instance, the category of endofunctions on Booleans.

This category is one which rises from a monoid, so we use the same definition previously seen to make this category in Agda.

First, we define the data types:

\begin{tabular}{lll}
\toprule
{\tt
\begin{lstlisting}[mathescape]
data Bool : Set where
  true  : Bool
  false : Bool

data Unit : Set where
  unit : Unit
\end{lstlisting}
}
\\
\bottomrule
\end{tabular}

Unit is the singleton set I will be using as the objects of the category of endofunctions on Booleans. Bool is the set of Booleans which I will be using to define all endofunctions.

With this in place, we can define the category:

\begin{tabular}{lll}
\toprule
{\tt
\begin{lstlisting}[mathescape]
set$\mathbb{B}$ : Cat
set$\mathbb{B}$ = record
        { Ob  = Unit
        ; Hom = $\lambda$ unit unit $\rightarrow$ Bool $\rightarrow$ Bool
        ; id  = $\lambda$ b $\rightarrow$ b
        ; _$\blacktriangleright$_ = $\lambda$ f g  $\rightarrow$ $\lambda$ x $\rightarrow$ g (f x)
        ; left_id  = refl
        ; right_id = refl
        ; assoc    = refl
        }
\end{lstlisting}
}
\\
\bottomrule
\end{tabular}

\section{\textbf{Set} in Agda}

With the previous definition, we were unable to define \textbf{Set}, however. To do this, we must define a universe level polymorphic record.

\begin{tabular}{lll}
\toprule
{\tt
\begin{lstlisting}[mathescape]
record Category {l : Level} : Set (lsuc l) where
  field Ob  : Set l
        Hom : Ob $\rightarrow$ Ob $\rightarrow$ Set l
        id  : {X : Ob} $\rightarrow$ Hom X X
        _$\blacktriangleright$_ : {X Y Z : Ob} $\rightarrow$ Hom X Y $\rightarrow$ Hom Y Z $\rightarrow$ Hom X Z

        left_id  : {X Y : Ob} {f : Hom X Y} $\rightarrow$ id $\blacktriangleright$ f $\equiv$ f
        right_id : {X Y : Ob} {f : Hom X Y} $\rightarrow$ f $\blacktriangleright$ id $\equiv$ f
        assoc    : {W X Y Z : Ob}
                   {f : Hom W X}
                   {g : Hom X Y}
                   {h : Hom Y Z} $\rightarrow$ f $\blacktriangleright$ (g $\blacktriangleright$ h) $\equiv$ (f $\blacktriangleright$ g) $\blacktriangleright$ h
\end{lstlisting}
}
\\
\bottomrule
\end{tabular}

With this definition, we can finally define \textbf{Set}:

\begin{tabular}{lll}
\toprule
{\tt
\begin{lstlisting}[mathescape]
set : Category
set = record
        { Ob  = Set
        ; Hom = $\lambda$ X Y $\rightarrow$ X $\rightarrow$ Y
        ; id  = $\lambda$ X $\rightarrow$ X
        ; _$\blacktriangleright$_ = $\lambda$ {X Y Z : Set} f g  $\rightarrow$ $\lambda$ x $\rightarrow$ g (f x)
        ; left_id  = refl
        ; right_id = refl
        ; assoc    = refl
        }
\end{lstlisting}
}
\\
\bottomrule
\end{tabular}

\section{Defining monads}

It is possible to define a monad in the traditional way:

\qquad
\begin{tabular}{lll}
  \tabitem T  & &an endofunctor on $C$ \\
  \tabitem $\eta$ & &a unit natural transformation\\
  \tabitem id  $\mu$ & &a multiplication natural transformation\\
\end{tabular}

The downside is that it would also require the definition of a functor and all the proofs associated. Alternatively, a monad can also be defined as a Kleisli Triple:

\qquad
\begin{tabular}{lll}
  \tabitem T  &: Ob C $\to$ Ob C &a morphism in C\\
  \tabitem $\eta$ &: {X : Ob C} $\to$ Hom$_C$(X,T X)&a unit transformation\\
  \tabitem $(\_)^*$ &: {X Y : Ob C} $\to$ Hom$_C$(X,T Y) $\to$ Hom$_C$(T X, T Y)&the Kleisli operator\\
\end{tabular}

This is easier to implement since no definition of a functor is required, and no mention of naturality is given either. Hence I implemented monads as a Kleisli triple.

Given the following projections:

\begin{tabular}{lll}
\toprule
{\tt
\begin{lstlisting}[mathescape]
Ob      = Category.Ob
Hom     = Category.Hom
id      = Category.id
compose = Category._$\blacktriangleright$_
\end{lstlisting}
}
\\
\bottomrule
\end{tabular}

The definition of a Kleisli Triple is given as follows:

\begin{tabular}{lll}
\toprule
{\tt
\begin{lstlisting}[mathescape]
record Monad {l : Level}{C : Category {l}} : Set (lsuc l) where
  field T  : Ob C $\to$ Ob C
        $\eta$   : {X : Ob C} $\to$ Hom C X (T X)
        _* : {X Y : Ob C} $\to$ Hom C X (T Y) $\to$ Hom C (T X) (T Y)

        monad_lid   : {X Y : Ob C}
                      {f : Hom C X (T Y)} $\to$
                      compose C $\eta$ (f *) $\equiv$ f
        monad_rid   : {X : Ob C} $\to$
                      ($\eta$ {X})* $\equiv$ id C {T X}
        monad_assoc : {X Y Z : Ob C}
                      {f : Hom C X (T Y)}
                      {g : Hom C Y (T Z)} $\to$
                      compose C (f *) (g *) $\equiv$ (compose C f (g *)) *
\end{lstlisting}
}
\\
\bottomrule
\end{tabular}

\section{State monad}

Now that monads are defined, it is possible to check the answer to the exercise at the end of the last chapter. We can show---in Agda---that the state monad defined previously is indeed a valid monad.

First, we need products defined:

\begin{tabular}{lll}
\toprule
{\tt
\begin{lstlisting}[mathescape]
data _$\times$_ (A B : Set) : Set where
  _,_ : A $\rightarrow$ B $\rightarrow$ A $\times$ B

$\pi_1$ : {A B : Set} $\rightarrow$ (A $\times$ B) $\rightarrow$ A
$\pi_1$ (a , b) = a

$\pi_2$ : {A B : Set} $\rightarrow$ (A $\times$ B) $\rightarrow$ B
$\pi_2$ (a , b) = b

$\pi$id : {A B : Set}{p : A $\times$ B} $\rightarrow$ $\pi_1$ p , $\pi_2$ p $\equiv$ p
$\pi$id {A}{B}{a , b} = refl
\end{lstlisting}
}
\\
\bottomrule
\end{tabular}

Assuming function extensionality:

\begin{tabular}{lll}
\toprule
{\tt
\begin{lstlisting}[mathescape]
postulate exten : {X Y : Set}{f g : X $\rightarrow$ Y} $\rightarrow$
                  ((x : X) $\rightarrow$ f x $\equiv$ g x) $\rightarrow$ (f $\equiv$ g)

exten2 : {X Y Z : Set}{f g : X $\rightarrow$ Y $\rightarrow$ Z} $\rightarrow$
         ((x : X)(y : Y) $\rightarrow$ f x y $\equiv$ g x y) $\rightarrow$ (f $\equiv$ g)
exten2 h = exten ($\lambda$ x $\rightarrow$ exten(h x))
\end{lstlisting}
}
\\
\bottomrule
\end{tabular}

We can now define the state monad---on \textbf{Set}---and prove its laws:

\begin{tabular}{lll}
\toprule
{\tt
\begin{lstlisting}[mathescape]
state : Set $\to$ Monad {lsuc lzero}{set}
state S = record
          { T  = $\lambda$ A $\to$ S $\to$ (S $\times$ A)
          ; $\eta$   = $\lambda$ a $\to$ $\lambda$ s $\to$ (s , a)
          ; _* = $\lambda$ g f $\to$ $\lambda$ s $\to$ g ($\pi_2$ (f s)) ($\pi_1$ (f s))
          ; monad_lid   = refl
          ; monad_rid   = lemma_rid
          ; monad_assoc = refl
          }
          where
            lemma_rid : {A : Ob set} $\to$
                        ($\lambda$f s $\to$ $\pi_1$ (f s) , $\pi_2$ (f s))
                        $\equiv$ id set {S $\to$ (S $\times$ A)}
            lemma_rid {A} =
              begin
                ($\lambda$ f s $\to$ $\pi_1$ (f s) , $\pi_2$ (f s))
                $\equiv$$\langle$ exten2 ($\lambda$ f s $\to$ $\pi$id {p = f s}) $\rangle$
                ($\lambda$ f s $\to$ f s)
                $\equiv$$\langle$ refl $\rangle$
                ($\lambda$ f $\to$ f)
              $\blacksquare$
\end{lstlisting}
}
\\
\bottomrule
\end{tabular}

\end{document}